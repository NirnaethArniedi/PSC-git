\setcounter{table}{0} \renewcommand{\thetable}{A.\arabic{table}} 

\begin{table}[h]
\centering
\begin{tabular}{|c|c|}	
\hline
\textbf{Distance aller parcourue (\si{km})} & \textbf{Proportion (\%{})} \\
\hline
$<2$ & 12,2 \\
$[2,5]$ & 15,9 \\
$[5,10]$ & 21,0 \\
$[10,20]$ & 23,9 \\
$[20,40]$ & 20,0 \\
$[40,80]$ & 5,9 \\
$>80$ & 1,0 \\
\hline
\end{tabular}
\caption{Distances aller des trajets domiciles-travail en France (source: Enquête nationale transports et déplacements 2008 \cite{enqueteGouv}). \label{distance}}
\end{table}

\begin{table}[h]
\centering
\begin{tabular}{|c||c|c|c|}
\hline
\textbf{Véhicule} & \textbf{Proportion} & \textbf{Taille de la batterie (kWh)} & \textbf{Autonomie en km} \\
\hline
Nissan Leaf  &  0.18   &  24  &  150 \\
Tesla S & 0.0161  & 60 & 370 \\
ZOE & 0.6389  & 22 & 125 \\
Smart ForTwo & 0.0573  & 17,6 & 100 \\
\hline
\end{tabular}
\caption{Caractéristiques des véhicules électriques considérés (source: \url{www.automobile-propre.com} \cite{donneesVehicule}, Wikipédia).\label{carVE}}
\end{table}

\begin{table}[h]
\centering
\begin{tabular}{|c|c||c|c|}
\hline
1h & 0,41\% & 13h & 0,68\%\\
2h & 0,41\% & 14h & 0,68\%\\
3h & 0,41\% & 15h & 0,68\%\\
4h & 0,41\% & 16h & 3,06\%\\
5h & 0,41\% & 17h & 3,06\%\\
6h & 0,00\% & 18h & 9,18\%\\
7h & 0,00\% & 19h & 25,51\%\\
8h & 0,41\% & 20h & 27,55\%\\
9h & 0,41\% & 21h & 15,31\%\\
10h & 0,41\% & 22h & 3,40\%\\
11h & 0,41\% & 23h & 3,40\%\\
12h & 0,41\% & 24h & 3,40\%\\
\hline
\end{tabular}
\caption{Répartition des horaires de retour du travail par tranche de \SI{1}{\hour} (source : Enquête trajets domicile-travail réalisée par l'ORSTIF en 2010 \cite{ORSTIF}).\label{horaires}}
\text{}
\end{table}
