\section*{Conclusion}
\addcontentsline{toc}{section}{Conclusion}

Nous avons donc vu tout au long de ce PSC comment prévoir l'impact de la recharge des véhicules électriques sur le réseau. Cette modélisation nous a poussé à faire des recherches bibliographiques approfondies et variées, et à prendre contact avec des entreprises telles que La Poste afin d'obtenir le plus d'informations possibles vis-à-vis des utilisateurs de ce produit ; qui reste encore aujourd'hui marginal. Nous avons poussé notre étude sur différents fronts afin d'avoir un modèle le plus complet possible. Notre modèle de diffusion permet d'obtenir une valeur du nombre de véhicules circulant en France en 2025. Ce chiffre de \num{163000} véhicules est à mettre en parallèle avec d'autres prévisions qui prévoyaient 1 millions de véhicules électriques en circulation. Nous avons de plus mené une étude fine du comportement des utilisateurs afin de produire un modèle robuste. Nous avons implémenté ce modèle en code informatique, relevant à ce moment là de nouveaux défis spécifiques à cette partie. Finalement les courbes de résultats obtenues sont en accords avec ce que l'on pourrait penser à première vue : la majorité de la recharge s'effectue le soir. Cependant nous voyons que si il existe des bornes sur le lieu de travail, on observe un deuxième pic de recharge vers 10 heures du matin. En outre la présence de bornes sur le lieu de travail ne semble pas diminuer le pic du soir : les utilisateurs ont tendance à se brancher dès qu'ils se garent. Une autre observation que nous pouvons faire est que l'implémentation du \smartgrid{} doit se faire de manière fine. En effet il faut que les véhicules dont on a stoppé la recharge pendant le pic de consommation ne soient pas tous remis à charger en même temps. Il faudrait donc définir un ordre de priorité pour la reprise de la charge des véhicules. Enfin la valeur du pic de consommation que l'on obtient, de l'ordre de \SI{80}{\mega\watt}, est à mettre en perspective avec la consommation française moyenne de \SI{80}{\giga\watt} en hiver. On voit donc que l'ordre de grandeur est d'un facteur \num{1000} lorsque l'on considère la flotte de véhicules que nous avons choisie pour notre modélisation. Ce sera donc à une entreprise telle que GDF Suez de décider si la mise en place du \smartgrid{} et du \emph{Vehicle to grid} est rentable.


