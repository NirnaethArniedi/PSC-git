\section*{Introduction}
\addcontentsline{toc}{section}{Introduction}

Nos habitudes de consommation d'énergie sont appelées à évoluer grandement dans les années à venir.
En effet la prise de conscience écologique, la fluctuation des cours du pétrole et la baisse du coût des technologies durables doivent nous amener à transformer notre comportement. 
Par exemple dans le domaine du transport, de nombreuses études sont menées depuis une vingtaine d'années pour trouver un remplaçant aux énergies fossiles, tel que les biocarburants ou les moteurs à hydrogène.
Au vu de l'orientation actuelle prise par les constructeurs il apparait que le véhicule électrique est le meilleur concurrent des véhicules thermiques. 
Afin d'anticiper cette profonde évolution du marché de l'énergie il faut prévoir l'impact qu'aura ce changement de demande énergétique du pétrole vers l'électricité. 

Faisons une expérience de pensée: supposons que les 38~millions de véhicules actuellement en France soient tous des véhicules électriques. Si toutes ces voitures se rechargent en même temps, en demandant une puissance de \SI{3.5}{\kilo\watt}, cela nécessite de produire une puissance totale de \SI{133}{\giga\watt}. Sachant qu'une centrale nucléaire ne délivre qu'une puissance d'un gigawatt, il faudra trouver d'autres solutions pour répondre au problème. C'est pour cela qu'il faut prévoir et modéliser l'impact qu'aura la recharge des véhicules électriques sur le réseau. Cela nécessite non seulement de prévoir le nombre de véhicules électriques qui seront en circulation dans vingt ans, mais aussi de modéliser le comportement des utilisateurs de véhicules électriques. 

En outre, on peut déjà supposer que cette recharge s'effectuera à partir de l'heure de retour du travail, entre 18~heures et 20~heures. Or c'est déjà l'heure du pic de consommation en France. Les fournisseurs d'énergie doivent donc augmenter la flexibilité de l'offre énergétique afin de s'adapter à cette nouvelle demande. Une des solutions peut être d'utiliser les batteries des véhicules électriques comme des \og{}mini-centrales\fg{}, capables lorsqu'elles sont branchées de fournir de l'énergie au réseau lorsque celui-ci en demande plus. On se place donc ici dans une problématique de \smartgrid{} (réseau intelligent) et d'agrégateurs de flexibilité. 

L'objectif de ce PSC est d'évaluer l'impact qu'aura la recharge des véhicules électriques sur le réseau dans quinze ans, et de voir quelles sont les solutions qui peuvent être utilisées dans le cadre du \smartgrid{}. Pour cela nous avons mené une étude sur le comportement des utilisateurs de véhicules, afin de connaitre leurs habitudes de recharge et de consommation, puis nous avons cherché à modéliser la diffusion des véhicules électriques en France afin d'avoir une idée de la taille du parc automobile en 2025. Enfin nous avons produit un programme informatique permettant de mettre en place le modèle et d'en obtenir des données sur la puissance demandée par la recharge des véhicules électriques.