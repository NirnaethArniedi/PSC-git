\section{Évolution du projet}

	Notre projet était au départ d'étudier l'usure des batteries d'appareils électriques, tels que les téléphones portables, au court de leurs cycles de charge/décharge. Nous avons ensuite pris contact avec l'entreprise GDF-Suez qui proposait de faire un PSC sur "L'étude d'un algorithme de charge/décharge des véhicules électriques". Cependant GDF-Suez nous a appris que le projet avait déjà été fait et nous a proposé un autre projet basé cette fois-ci sur une étude plus orientée sur le comportement des utilisateurs de véhicules électriques. C'est ainsi que nous avons entamé ce PSC sur la modélisation de l'impact de la recharge des véhicules électriques sur le réseau. 
	
	Notre planning prévisionnel était de faire de la recherche de données sur le comportement des utilisateurs et sur les caractéristiques des véhicules électriques jusqu'à fin décembre. Ensuite nous devions entamer la partie modélisation et implémentation du code.
	
	Concrètement le planning prévisionnel à été bien tenu. La recherche de données et l'exploitation de ces données nous a demandé beaucoup de travail au début du PSC. En effet comme le véhicule électrique est encore un marché de niche il existe très peu d'études sur le comportement de leurs utilisateurs. Nous avons donc prospecté quelques études faites sur des petits échantillons et nous avons extrapolées des données sur les utilisateurs de véhicules thermiques, en considérant que leurs comportements varient peu avec les utilisateurs de véhicules électriques (pour les horaires de départ et de retour du travail et les distances parcourues par exemple).
	
	Parallèlement à cette recherche documentaire nous nous sommes penchés sur la forme qu'aura le résultat produit par notre modèle. Nous avons donc opté pour une courbe de la puissance consommée par la recharge des véhicules électriques en fonction de l'heure de la journée. 
	Nous avons commencé la partie modélisation au mois de janvier. Nous pensions pouvoir la terminer plus rapidement, mais en réalité nous avons rencontré des difficultés nouvelles. En effet il y avait de nombreux paramètres qui dépendaient les uns des autres, et produire un modèle mathématiques pour définir ces interactions n'était pas évident.
	Nous avons produit un premier programme informatique basique pour le rapport intermédiaire. Celui-ci nous a permis de tester rapidement notre modèle, mais était limité dans les modifications que l'on pouvait lui apporter, notamment au niveau du \smartgrid{} ou du comportement des utilisateurs.
	Nous nous sommes alors lancés dans une partie purement algorithmique, afin de produire un programme qui soit rapide à exécuter pour de nombreux véhicules et facilement modifiable. Le travail en groupe était plus difficile à mettre en place lors de ce moment car nous ne possédions pas tous les connaissances requises en informatique.
	
	c'est pour cela qu'une autre partie du groupe s'est penché sur une nouvelle problématique apportée par le PSC : la prévision du nombre de véhicules électriques dans une quinzaines d'années. Ce n'était pas une partie que nous avions mis dans notre planning de départ mais elle nous permet d'apporter des éléments concrets pour les données d'entrées de notre modèle.
	